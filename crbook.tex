\documentclass{crbook}

\begin{document}
\ExplSyntaxOn

% First page
\title{Competitive\ Rust\ Book\ in\ Simplified\ Chinese}
\author{LittleYe233}
\l_crb_datemodified:n {\today}

\g_crb_maketitle

% Other pages
\newpage
\ExplSyntaxOff

\section{文件操作}
\subsection{读写到标准输入输出 (Codeforces)}

Codeforces 等平台要求程序读写重定向到标准输入输出. 以下模板使用了一个自定义结构体 \texttt{Scanner} 实现按 token (以 whitespaces 隔开) 每次读入一个数据, 涵盖大部分基础类型和 \texttt{String} 等类; 输出则使用 \texttt{std::io::BufWriter} 加快写入速度.

\begin{listing}
    \linespread{1}
    \begin{minted}[linenos, breaklines, frame=lines]{rust}
#[allow(unused_imports)]
use std::io::{BufWriter, stdin, stdout, Write}

#[derive(Default)]
struct Scanner {
    buffer: Vec<String>
}
impl Scanner {
    fn next<T: std::str::FromStr>(&mut self) -> T {
        loop {
            if let Some(token) = self.buffer.pop() {
                return token.parse().ok().expect("Failed parse");
            }
            let mut input = String::new();
            stdin().read_line(&mut input).expect("Failed read");
            self.buffer = input.split_whitespace().rev().map(String::from).collect();
        }
    }
}

fn main() {
    let mut scan = Scanner::default();
    let out = &mut BufWriter::new(stdout());
    // Write code here
}
    \end{minted}
    \caption{读写到标准输入输出模板}
\end{listing}

以下为输入输出的例子:

\begin{listing}
    \linespread{1}
    \begin{minted}[linenos, breaklines, frame=lines]{rust}
let a: i32 = scan.next();
let b = scan.next::<u64>();
writeln!(out, "{} {}", a, b).ok();
    \end{minted}
    \caption{具体的输入输出示例}
\end{listing}

\section{算术运算}
\subsection{判断浮点数的值是否为整数}

\texttt{f32} 和 \texttt{f64} 类型实现了 \texttt{.fract()} 方法, 用以输出浮点数的小数部分. 官方示例如下\footnote{\url{https://doc.rust-lang.org/std/primitive.f64.html\#method.fract}}:

\begin{listing}
    \linespread{1}
    \begin{minted}[linenos, breaklines, frame=lines]{rust}
let x = 3.6_f64;
let y = -3.6_f64;
let abs_difference_x = (x.fract() - 0.6).abs();
let abs_difference_y = (y.fract() - (-0.6)).abs();

assert!(abs_difference_x < 1e-10);
assert!(abs_difference_y < 1e-10);
    \end{minted}
    \caption{.fract() 方法示例}
\end{listing}

因此, 只需要判断这个方法的返回值是否为 \texttt{0.0} 即可判断浮点数的值是否为整数.

\newpage
\section{附录}
\subsection{程序清单列表}
% See: https://tex.stackexchange.com/a/266903
\makeatletter
\@starttoc{lol}
\makeatother

\end{document}
