\documentclass{crbook}

\begin{document}
\ExplSyntaxOn

% First page
\title{Competitive\ Rust\ Book\ in\ Simplified\ Chinese}
\author{LittleYe233}
\l_crb_datemodified:n {\today}

\g_crb_maketitle

% Other pages
\newpage
\ExplSyntaxOff

\section{文件操作}
\subsection{读写到标准输入输出 (Codeforces)}

Codeforces 等平台要求程序读写重定向到标准输入输出. 以下模板使用了一个自定义结构体 \texttt{Scanner} 实现按 token (以 whitespaces 隔开) 每次读入一个数据, 涵盖大部分基础类型和 \texttt{String} 等类; 输出则使用 \texttt{std::io::BufWriter} 加快写入速度.

\begin{listing}
    \linespread{1}
    \begin{minted}[linenos, breaklines, frame=lines]{rust}
#[allow(unused_imports)]
use std::io::{BufWriter, stdin, stdout, Write}

#[derive(Default)]
struct Scanner {
    buffer: Vec<String>
}
impl Scanner {
    fn next<T: std::str::FromStr>(&mut self) -> T {
        loop {
            if let Some(token) = self.buffer.pop() {
                return token.parse().ok().expect("Failed parse");
            }
            let mut input = String::new();
            stdin().read_line(&mut input).expect("Failed read");
            self.buffer = input.split_whitespace().rev().map(String::from).collect();
        }
    }
}

fn main() {
    let mut scan = Scanner::default();
    let out = &mut BufWriter::new(stdout());
    // Write code here
}
    \end{minted}
    \caption{读写到标准输入输出模板}
\end{listing}

以下为输入输出的例子:

\begin{listing}
    \linespread{1}
    \begin{minted}[linenos, breaklines, frame=lines]{rust}
let a: i32 = scan.next();
let b = scan.next::<u64>();
writeln!(out, "{} {}", a, b).ok();
    \end{minted}
    \caption{具体的输入输出示例}
\end{listing}

\newpage
\section{算术运算}
\subsection{判断浮点数的值是否为整数}

\texttt{f32} 和 \texttt{f64} 类型实现了 \texttt{.fract()} 方法, 用以输出浮点数的小数部分. 官方示例如下\footnote{\url{https://doc.rust-lang.org/std/primitive.f64.html\#method.fract}.}:

\begin{listing}
    \linespread{1}
    \begin{minted}[linenos, breaklines, frame=lines]{rust}
let x = 3.6_f64;
let y = -3.6_f64;
let abs_difference_x = (x.fract() - 0.6).abs();
let abs_difference_y = (y.fract() - (-0.6)).abs();

assert!(abs_difference_x < 1e-10);
assert!(abs_difference_y < 1e-10);
    \end{minted}
    \caption{.fract() 方法示例}
\end{listing}

因此, 只需要判断这个方法的返回值是否为 \texttt{0.0} 即可判断浮点数的值是否为整数.

\subsection{最值问题}

Rust 中得到一个切片 \texttt{v} 的最值 (以最大值为例) 只需使用 \texttt{*v.iter().max().unwrap()} 即可. 若要得到该最值对应的索引 (类似于 NumPy 中的 \texttt{.argmax()}), 会稍显麻烦一些\footnote{\url{https://stackoverflow.com/a/57815298/12002560}.}:

\begin{listing}
    \linespread{1}
    \begin{minted}[linenos, breaklines, frame=lines]{rust}
fn argmax<T: Ord + Copy>(v: &[T]) -> Option<usize> {
    v.iter().enumerate().max_by_key(|(_, &v)| v).map(|(idx, _)| idx)
}

fn argmax<T: Ord>(v: &[T]) -> Option<usize> {
    v.iter().enumerate().max_by(|(_, v0), (_, v1)| v0.cmp(v1)).map(|(idx, _)| idx)
}
    \end{minted}
    \caption{NumPy 中 .argmax() 的部分实现}
\end{listing}

\noindent 两个版本分别适用于实现与否 \texttt{Copy} trait 的情况. 示例如下:

\begin{listing}
    \linespread{1}
    \begin{minted}[linenos, breaklines, frame=lines]{rust}
fn argmax<T: Ord + Copy>(v: &[T]) -> Option<usize> {
    v.iter().enumerate().max_by_key(|(_, &v)| v).map(|(idx, _)| idx)
}

fn main() {
    let v = vec![100, 90, 1000, -200, 1, -1000, 800];
    assert_eq!(*v.iter().max().unwrap(), 1000);
    assert_eq!(argmax(&v).unwrap(), 2)
}
    \end{minted}
    \caption{argmax() 的实际应用}
\end{listing}

\subsection{上下取整和四舍五入}

上下取整和四舍五入的需求在编程竞赛中经常出现. 对于整数相除的上下取整需求, 如果仅考虑整数运算, 可以使用如下的公式:
\begin{gather*}
    \mathrm{floor}\left(\frac{a}{b}\right)=a/b, \\
    \mathrm{ceil}\left(\frac{a}{b}\right)=(a+b-1)/b.
\end{gather*}
其中斜杠 ($/$) 表示编程语言中的整数除法.

对于浮点数的四舍五入运算, 可以使用如下的公式:
\begin{equation*}
    \mathrm{round}(x)=\mathrm{floor}(x+0.5).
\end{equation*}

参考如下的例子:
\begin{listing}
    \linespread{1}
    \begin{minted}[linenos, breaklines, frame=lines]{rust}
fn main() {
    let (a, b) = (23, 4);
    assert_eq!(a / b, (((a as f64) / (b as f64)).floor() as i32));
    assert_eq!((a + b - 1) / b, (((a as f64) / (b as f64)).ceil() as i32));
    
    let (x, y, z) = (3.5_f64, 3.1_f64, -3.5_f64);
    assert_eq!((x + 0.5).floor(), 4.0);
    assert_eq!((y + 0.5).floor(), 3.0);
    assert_eq!((z + 0.5).floor(), -3.0);
}
    \end{minted}
    \caption{上下取整和四舍五入的示例}
\end{listing}


\newpage
\section{算法}
\subsection{二分查找}

我们直接考虑实现 C++ 中的 \texttt{std::lower\_bound()} 和 \texttt{std::upper\_bound()}. 这两个函数的含义分别是\footnote{\url{https://en.cppreference.com/w/cpp/algorithm/lower\_bound} 和 \url{https://en.cppreference.com/w/cpp/algorithm/upper\_bound}.}:

\begin{itemize}
    \item \texttt{std::lower\_bound()}: 返回第一个\textbf{不在}指定值\textbf{前}的元素;
    \item \texttt{std::upper\_bound()}: 返回第一个\textbf{在}指定值\textbf{后}的元素.
\end{itemize}

这可以直接由 Rust 中的 \texttt{.binary\_search\_by()} 实现\footnote{\url{https://stackoverflow.com/a/75790348/12002560}.}. 假定需要二分查找的切片为升序排列的 \texttt{Vec<i32>} (也就是说, \textit{限定了其容器类型和内容物类型}).

\begin{listing}
    \linespread{1}
    \begin{minted}[linenos, breaklines, frame=lines]{rust}
fn upper_bound(v: &Vec<i32>, x: i32) -> usize {
    v.binary_search_by(|probe| match probe.cmp(&x) {
        Ordering::Equal => Ordering::Less,
        ord => ord
    }).unwrap_err()
}
    \end{minted}
    \caption{C++ std::upper\_bound() 函数的实现}
\end{listing}

若实现 \texttt{std::lower\_bound()}, 只需将第 3 行的 \texttt{Ordering:Less} 改为 \texttt{Ordering:Greater} 即可. 以下为该实现实际使用的示例:

\begin{listing}
    \linespread{1}
    \begin{minted}[linenos, breaklines, frame=lines]{rust}
use core::cmp::Ordering;

fn upper_bound(v: &Vec<i32>, x: i32) -> usize {
    v.binary_search_by(|probe| match probe.cmp(&x) {
        Ordering::Equal => Ordering::Less,
        ord => ord
    }).unwrap_err()
}

fn lower_bound(v: &Vec<i32>, x: i32) -> usize {
    v.binary_search_by(|probe| match probe.cmp(&x) {
        Ordering::Equal => Ordering::Greater,
        ord => ord
    }).unwrap_err()
}

fn main() {
    let v = vec![-100, -99, -1, 0, 8, 8, 100];
    assert_eq!(lower_bound(&v, 7), 4);
    assert_eq!(lower_bound(&v, 8), 4);
    assert_eq!(lower_bound(&v, 9), 6);
    assert_eq!(upper_bound(&v, 7), 4);
    assert_eq!(upper_bound(&v, 8), 6);
    assert_eq!(upper_bound(&v, 9), 6);
}
    \end{minted}
    \caption{std::lower\_bound() 和 std::upper\_bound() 在 Rust 中实现的实际应用}
\end{listing}

\newpage
\section{数据结构}
\subsection{哈希表 (Hash Map)}

\subsection{哈希集 (Hash Set)}

% Appendix
\newpage
\section{附录}
\subsection{程序清单列表}
% See: https://tex.stackexchange.com/a/266903
\makeatletter
\@starttoc{lol}
\makeatother

\end{document}
